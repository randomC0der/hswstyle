%Copyright (c) 2019-2021 Lukas Hein

% Nur der Nachname in der Formatierung
\DeclareNameFormat{lastname}{\usebibmacro{name:family}{\namepartfamily}{}{}{}\usebibmacro{name:andothers}}

% Autoren formatieren
\DeclareDelimFormat*{multinamedelim}{/}
\DeclareDelimFormat*{finalnamedelim}{/}
\DeclareNameAlias{sortname}{family-given}
\DefineBibliographyStrings{german}{
    andothers = {et al.},
    nodate = {o.J.},
    bibliography = {Quellenverzeichnis}
}
\DefineBibliographyStrings{english}{
    andothers = {et al.},
}

% Formatierung entfernen
\DeclareFieldFormat[book,incollection,article,online,report]{title}{#1}
\DeclareFieldFormat{journal}{#1}
\DeclareFieldFormat{journaltitle}{#1}
\DeclareFieldFormat{booktitle}{#1}
\DeclareFieldFormat{url}{#1}
\DeclareFieldFormat{urldate}{#1}

% für einfache Quellenangaben
\DeclareCiteCommand{\hswcite}[\mkbibfootnote]
{\quad\usebibmacro{prenote}} % z.B. Vgl.
{\usebibmacro{citeindex}\usebibmacro{shortauthor}} % z.B. Mustermann (2016)
{} % multicitedelim
{\usebibmacro{postnote}} % Seitenzahl

% Wird für die Fußnoten für Bilder verwendet 
\DeclareCiteCommand{\hswfncite}[\mkbibfootnotetext]
{\quad\usebibmacro{prenote}} % z.B. Vgl.
{\usebibmacro{citeindex}\usebibmacro{shortauthor}} % z.B. Mustermann (2016)
{} % multicitedelim
{\usebibmacro{postnote}} % Seitenzahl

% für mehrere Quellenangaben in einer Fußnote
\DeclareMultiCiteCommand{\hswcites}[\mkbibfootnote]
{\hswcite}{\multicitedelim}

\DeclareMultiCiteCommand{\hswfncites}[\mkbibfootnotetext]
{\hswfncite}{\multicitedelim}

% bildet Monographie ab
\DeclareBibliographyDriver{book}{%
    \usebibmacro{author}:\\%
        \failsafePrintfield{title}, \iffieldundef{edition}{\mkbibordedition{1}~\bibstring{edition}}{\printfield{edition}}, \failsafePrintlist{publisher}.\newline%
}

% bildet Sammelwerkbeiträge ab
\DeclareBibliographyDriver{incollection}{%
    \usebibmacro{author}:\\%
        \failsafePrintfield{title}; in: \usebibmacro{editor}, \failsafePrintfield{booktitle}, \failsafePrintlist{publisher}, \failsafePrintfield{pages}.\newline%
}

% bildet Zeitschriftenartikel ab
\DeclareBibliographyDriver{article}{%
    \usebibmacro{author}:\\%
        \failsafePrintfield{title}; in: \usebibmacro{journal}, \iffieldundef{volume}{o.Jg.}{\printfield{volume}. Jahrgang}, Heft: \failsafePrintfield{number}, \failsafePrintfield{pages}.\newline%
}

% bildet Internetquellen ab
\DeclareBibliographyDriver{online}{%
    \usebibmacro{author}:\\%
        \failsafePrintfield{title}, \failsafePrintfield{url}, Stand: \iffieldundef{urlyear}{\today}{\printurldate}.\newline%
}

% bildet öffentliche betriebliche Quellen ab
\DeclareBibliographyDriver{report}{%
    \usebibmacro{author}:\\%
        \failsafePrintfield{title}.\newline%
}
